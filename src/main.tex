\documentclass[10pt, preprint]{sigplanconf}

\usepackage{amsmath}
\usepackage{url}
\usepackage[usenames,dvipsnames]{color}
\usepackage[english]{babel}
\usepackage{graphicx}
\usepackage{listings}
\usepackage[compact]{titlesec}
\usepackage{multirow}
\usepackage{color, colortbl}
\usepackage{fancyvrb}
\usepackage{shortvrb}
\usepackage{setspace}
%\usepackage{flushend}
\usepackage{stfloats}
\usepackage{listings,xcolor,beramono}
\usepackage{tikz}
\usepackage{calc}
\usepackage[utf8x]{inputenc}
%% typesetting type system rules:
\usepackage{latexsym}
\usepackage{bcprules}
\usepackage[T1]{fontenc}

% Short verbatim
\DefineShortVerb{\|}
\fvset{fontsize=\small}

\definecolor{Gray}{gray}{0.9}
\definecolor{LightGray}{gray}{0.4}
\newcolumntype{g}{>{\columncolor{Gray}}r}

\newcommand\Small{\fontsize{8.5}{8.5}\selectfont}
\newcommand*\LSTfont{\Small\ttfamily\lsstyle}

% Divisible by
\DeclareRobustCommand{\divby}{%
  \mathrel{\vbox{\baselineskip.65ex\lineskiplimit0pt\hbox{.}\hbox{.}\hbox{.}}}%
}

\makeatletter
\newenvironment{btHighlight}[1][]
{\begingroup\tikzset{bt@Highlight@par/.style={#1}}\begin{lrbox}{\@tempboxa}}
{\end{lrbox}\bt@HL@box[bt@Highlight@par]{\@tempboxa}\endgroup}

\newcommand\btHL[1][]{%
  \begin{btHighlight}[#1]\bgroup\aftergroup\bt@HL@endenv%
}
\def\bt@HL@endenv{%
  \end{btHighlight}%
  \egroup
}
\newcommand{\bt@HL@box}[2][]{%
  \tikz[#1]{%
    \pgfpathrectangle{\pgfpoint{1pt}{0pt}}{\pgfpoint{\wd #2}{\ht #2}}%
    \pgfusepath{use as bounding box}%
    \node[anchor=base west, fill=black!10,outer sep=0pt,inner xsep=1pt, inner ysep=0pt, rounded corners=1pt, minimum height=\ht\strutbox+1pt,#1]{\raisebox{1pt}{\strut}\strut\usebox{#2}};
  }%
}

\g@addto@macro\normalsize{%
  \setlength\abovedisplayskip{-0.5em}
  \setlength\belowdisplayskip{1em}
  \setlength\abovedisplayshortskip{0.25em}
  \setlength\belowdisplayshortskip{0.25em}
}
\makeatother

% "define" Scala
\lstdefinelanguage{scala}{
  morekeywords={abstract,case,catch,class,def,%
    do,else,extends,false,final,finally,%
    for,if,implicit,import,match,mixin,%
    new,null,object,override,package,%
    private,protected,requires,return,sealed,%
    super,this,throw,trait,true,try,%
    type,val,var,while,with,yield},
  otherkeywords={=>,<-,<\%,<:,>:,\#,@},
  sensitive=true,
  morecomment=[l]{//},
  morecomment=[n]{/*}{*/},
  morestring=[b]",
  morestring=[b]',
  morestring=[b]""",
  moredelim=**[is][\btHL]{`}{`},
}

% Default settings for code listings
%\lstset{language=scala,showstringspaces=false,columns=flexible, basicstyle=\footnotesize \ttfamily}
\definecolor{ggray}{gray}{0.5}
\renewcommand{\ttdefault}{pcr}
\lstset{frame=tb,
  language=scala,
  aboveskip=3mm,
  belowskip=3mm,
  showstringspaces=false,
  columns=flexible,
  basicstyle={\footnotesize \ttfamily},
  %basicstyle=\LSTfont,
%% numbers:
%  numbers=none,
  numbers=left,
  %xleftmargin=2em,
  %framexleftmargin=1.5em,
%%% end numbers
  numberstyle=\tiny\color{gray},
  keywordstyle=\bfseries,
  commentstyle=\em\color{ggray},
  stringstyle=\em,
%  keywordstyle=\color{blue},
%  commentstyle=\color{OliveGreen},
%  stringstyle=\color{Purple},
  frame=single,
  breaklines=true,
  breakatwhitespace=true
  tabsize=3
}

%% Line numbers inside frame:
%% http://tex.stackexchange.com/questions/30504/how-can-i-properly-align-the-line-numbers-of-a-source-code-listing-with-the-marg
\makeatletter
\newlength{\linenumwidth} \setlength{\linenumwidth}{2em}% Redefine as required
\newlength{\numwidth}%
\setlength{\numwidth}{\widthof{\normalfont{\lst@numberstyle{9}}}}% Up to 2-digit (99) line numbers
\def\lst@PlaceNumber{%
  \makebox[\numwidth+0.3em][l]{%
    \makebox[\numwidth][r]{\normalfont\lst@numberstyle{\thelstnumber}}%
  }%
}
\makeatother


%%% How to prevent lstlisting from splitting code between pages?
%%% http://tex.stackexchange.com/questions/10141/how-to-prevent-lstlisting-from-splitting-code-between-pages
\lstnewenvironment{lstlisting-nobreak}[1][]%
{
   \noindent
   \vspace{-0.1em}
   \minipage{\linewidth}
   \vspace{0.12\baselineskip}
%    \lstset{basicstyle=\ttfamily\footnotesize,frame=single,#1}
}
{
   \vspace{-0.1em}
   \vspace{-0.18\baselineskip}
   \endminipage
}

% Footnote remember - http://anthony.liekens.net/index.php/LaTeX/MultipleFootnoteReferences
\newcommand{\footnoteremember}[2]{
\footnote{#2}
\newcounter{#1}
\setcounter{#1}{\value{footnote}}
}
\newcommand{\footnoterecall}[1]{
\footnotemark[\value{#1}]
}
% use:
% \footnoteremember{myfootnote}{This is my footnote}
% \footnoterecall{myfootnote}

% Packed item, so we don't waste space
\newenvironment{packed_item}{
\begin{itemize}
  \setlength{\itemsep}{1pt}
  \setlength{\parskip}{0.2pt}
  \setlength{\parsep}{0.2pt}
}{\end{itemize}}

% Packed enum, so we don't waste space
\newenvironment{packed_enum}{
\begin{enumerate}
  \setlength{\itemsep}{1pt}
  \setlength{\parskip}{0.2pt}
  \setlength{\parsep}{0.2pt}
}{\end{enumerate}}

% Review tools
%\newcommand{\topic}[1]{#1}
\newcommand{\topic}[1]{{\bf \color{Bittersweet}#1}}
% This marks comments
%\newcommand{\vu}[1]{{\color{BurntOrange}\framebox{{\bf VU}}\;#1}\;}
%\newcommand{\tr}[1]{{\color{Blue}{\bf \framebox{TR}\;#1}}\;}
%\newcommand{\as}[1]{{\color{Red}{\bf \framebox{AS}\;#1}}\;}
%\newcommand{\mo}[1]{{\color{OliveGreen}{\bf \framebox{MO}\;#1}}\;}
\newcommand{\textem}[1]{{\em#1}}
\newcommand{\newterm}[1]{{\em #1}}

% Paper HACKS
\setstretch{0.97}
\renewcommand{\bibfont}{\scriptsize}
%\renewcommand{\UrlFont}{\scriptsize}
\renewcommand*{\UrlFont}{\ttfamily\footnotesize\relax}

\begin{document}

\clubpenalty=10000
\widowpenalty = 10000

\titlebanner{DRAFT - Please do not circulate}        % These are ignored unless
\preprintfooter{DRAFT - Please do not circulate}     % 'preprint' option specified.

\title{Improving the Performance of Scala Collections with Miniboxing}

\authorinfo{Aymeric Gen\^et \and Vlad Ureche \and Martin Odersky}
           {EPFL, Switzerland}
           {\{first.last\}@epfl.ch}

\maketitle

Using generics, Scala collections can be used to store different types of data in a type-safe manner. Unfortunately, due to the erasure transformation, the performance of generics is degraded when storing primitive types, such as integers and floating point numbers. Miniboxing \cite{miniboxing} is a novel translation for generics that restores primitive type performance. Naturally, a good choice would be to use miniboxing to translate Scala collections. In this paper we explore the patterns used to implement the Scala collections, describe how they are transformed by miniboxing and finally compare the performance of the two transformations on a mockup of the Scala collection library. The benchmarks show our prototype implementation\footnote{\url{http://scala-miniboxing.org}} can speed up collection operations by 45\% without any need for programmer intervention.

% \category{E.2}{Object representation}{}

% \keywords
% specialization; miniboxing; scala collections; hashmap; builder pattern; type classes; numeric pattern; macros; quasiquotes; Scala

\section{Introduction}

\topic{Scala collections allow storing data} in an abstract and type-safe manner. This is done using generics, which allow treating types as parameters of classes and methods. Using generics, it is possible to abstract over the type of the data in a collection, such as, for example, creating a linked list of integers. Safety is then guaranteed by the type system, which can guarantee all elements of the collection are actually integers. This increases productivity and improves code quality.

Generics are currently translated to low level bytecode using the technique of erasure \cite{java-erasure}. This entails that all type parameters are replaced by their lower bound, which is usually |Object|. As an advantage, it makes everything uniform, since everything is treated as a reference. However, since generics are represented as references, the type parameters cannot be directly instantiated by primitve types, such as integers or floating point numbers. Instead, an object representation of the primitive type must be used. This is done by wrapping primitive values into objects, in a process is called boxing. The opposite process, which extracts the primitive value from an object, is called unboxing. Boxing and unboxing degrade program performance, inflate the heap memory requirements and trigger extra garbage collection cycles.

\topic{Miniboxing \cite{miniboxing} is an alternative translation for generics,} which avoids boxing and unboxing, thus improving the performance for primitive types. This is done by creating additional versions of the generic methods and classes which accept primitive values as arguments and return primitive types. Instead of creating an additional version for each primitive type, which would be wasteful in terms of bytecode size, miniboxing creates a single version which can encode all primitive types. We call these additional versions of methods and classes specialized variants. Therefore, in the case of miniboxing, for a single type parameter, there will be two variants of a method or class: one using the erasure-based translation, which is used for objects, and a specialized one, for primitive types.

\topic{Scala collections expose a simple and high-level interface.} This is done by allowing programmers to effortlessly transform collections by mapping over their elements, filtering them or splitting collections based on custom criteria. All these features make heavy use of generics and are thus affected by slowdons when used with primitive types. This makes Scala collections unsuitable for numeric processing applications, such as machine learning or bioinformatics.

\topic{This leads to the idea of translating Scala collections} using the miniboxing transformation in order to reap the benefits of the convenient high-level interface while offering good performance for numeric applications. Yet this is not an easy task: collections are implemented using multiple layers of functionality and use complex patterns in order to reduce code duplication and gain flexibility.

\topic{In this paper, we set out to use the miniboxing transformation on a mock-up of the Scala collections,} which includes all the relevant patterns used in the real collections. In this context, we make the following contributions:

\begin{packed_item}
\item explaining the patterns that implement Scala collections;
\item showing how the miniboxing transforms each pattern;
\item benchmarking our mock-up of the Scala collections using both the erasure and miniboxing transformations.
\end{packed_item}

The paper first describes the miniboxing transformation, implemented as a plugin for the Scala compiler, then describes multiple patterns used in implementing Scala collections and their transformation with the miniboxing plugin: (1) their hierarchy, (2) the closures, (3) the builder Pattern, and (4) the Numeric pattern. Finally, it presents our mockup collections and the benchmark results.
\section{Miniboxing}

Miniboxing\cite{miniboxing-www} is a compilation scheme that improves the performance of generics in the Scala programming language. The miniboxing transformation is activated by annotating type parameters with |@miniboxed|, for both classes and methods:

\begin{lstlisting-nobreak}
class C[`@miniboxed T`](var t: T) {
  def foo(): T = t
}
\end{lstlisting-nobreak}

In order to understand the transformation behind this annotation, let's look at the following example. Assume we want to write a generic method that is likely to use a primitive type parameter:

\begin{lstlisting-nobreak}
def bar[`@miniboxed T`](t: T): T =
  (new C[T](t)).foo()
\end{lstlisting-nobreak}

Since the type parameter |T| of foo is marked as |@miniboxed|, the method will have two versions: an optimized version for value types, and the compatible, erasure-based, slow version of the method:

\begin{lstlisting-nobreak}
def bar_J(`T_Tag: byte`, t: `long`): `long` =
  (new C_J(t)).foo_J()   // optimized version
def bar(t: `Object`): `Object` =
  (new C(t)).foo()   // erasure-based version
\end{lstlisting-nobreak}

Every time the method |bar[T]| is called with a primitive type parameter, the optimized version |bar_J| will be used instead. However, when the method is called with an object type parameter, such as a |String|, the original one will be used:

\begin{lstlisting-nobreak}
bar("x")  // bar() is used
bar(3)      // bar_J() is used
\end{lstlisting-nobreak}

Now let's look at the miniboxed class |C| from the previous example. The same kind of transformation as the one for methods will occur, two representations of the same class will be made:

\begin{lstlisting-nobreak}
class C_J(`T_Tag: byte`, t: `long`) {
  def foo_J(): `long` = t
  def foo(): `long` = foo_J()
}

class C(t: `Object`) {
  def foo_J(): `Object` = foo()
  def foo(): `Object` = t
}
\end{lstlisting-nobreak}

Once again, if the class |C[T]| is instantiated with a primitive type paramater, it's |C_J| that will take place instead, otherwise, it's the original |C| class that takes place. As one can see, the methods inside the class are wired in order to match the correct call. If one uses the method |foo| in the optimized |C_J|, it's actually the |foo_J| method that must be called:

\begin{lstlisting-nobreak}
val c_s = new C[String]("x")  // class C is used
val c_i = new C[Int](3)               // class C_J is used
println(c_s.foo())   // foo() is used
println(c_i.foo())   // foo_J() is used
\end{lstlisting-nobreak}

The process of specialization on classes brings two important things. First, it will specialize the fields of the class. This will improve performance, because the program now deals with a shorter representation of the values. This is done by referring the fields as long integers, instead of objects. The conversion between the actual primitive type and the long value stored will also be made in the backend.

Second, it will specialize the methods and their bodies in the class, just like stand-alone methods. This will also greatly improve performance. By specializing them, methods receive and return long integers, instead of references. Thanks to this, boxing and unboxing, originally occurring with references, will be avoided, because the methods only deal with primitive representation of the values. Once again, conversion between the actual primitive type and the long value stored will be made in a hidden way.
\section{Scala Collections}

% \topic{Scala collections provide a high-level interface but lack in performance.}

\topic{In this paper we show how the miniboxing transformation enables improved collections}, which expose the same high-level interface without sacrificing performance. The next section presents the common patterns that enable the high-level interface in the Scala collections \cite{collections-arch}, and how miniboxing can be applied in order to improve performance.

%\topic{In this paper we show how the miniboxing transformation improves collection performance} while exposing the same high-level interface.
%Before we dive into optimizing, we will present the main patterns used to implement collections in Scala and how each of these patterns are transformed by the miniboxing transformation.
% Note that we will present only the most common patterns.
%Scala Collections use different patterns to expose their high-level interface consistently across all classes.

\subsection{Inheritance and Mixins}

\topic{Inheritance and mixins group the common behavior of different collections.} This reduces code duplication and gives rise to a convenient collection hierarchy, where each level of the inheritance makes more assumptions about the architecture than the previous level. For example, the path to a linked list goes through |Traversable|, |Iterable|, |Seq|, |LinearSeq| and finally |List|. % Each level makes an additional assumption: that the collection can be traversed, that it can be iterated, that it is a sequence, a linear sequence, and finally that it is a list.

This nesting and splitting of functionality makes is necessary to have deep miniboxing: Adding the |@miniboxed| annotation to a collection type parameter will not be enough to fully transform it, as most of its functionality will be inherited from parent traits. Instead, what needs to be done is to deeply visit all the parent traits and mark their arguments as |@miniboxed|:

\begin{lstlisting-nobreak}
 // trait/class definition needs to be marked:
 trait Traversable[`@miniboxed` +A] extends
 // parents' definitions also have to be marked:
       TraversableLike[`A`, Traversable[A]]
       with GenTraversable[`A`]
       with TraversableOnce[`A`]
       with GenericTraversableTemplate[`A`, Traversable] { ... }
\end{lstlisting-nobreak}

%Since the elements of the CollectionLike can be primitive types, the methods implemented can receive or return boxes of primitive values, which degrades performance. Thus, triggering the miniboxing plugin on the type parameter of the element offers better performance. However, we have to annote only the generic type of the elements contained in the collection, and not the type constructor. Indeed, since this is an abstraction over a container that will always be an object, there is no need to make a primitive type optimized version for it. Therefore, we get better performance if we use a custom CollectionLike with a |@miniboxed| notation on the type |T|:

Since the goal of Scala collections is to avoid code duplication, collection comprehensions, such as |map| and |filter|, all rely on a common mechanism: visiting each element in the collection, performing an action for it, and optionally adding a new element to the result collection. For example, |filter| visits all elements and for each element applies a predicate which decides whether the element should be part of the resulting collection or not.

The two key elements necessary for implementing collection comprehensions are: (1) the mechanism to visit each element using a custom function, which is implemented in |Traversable| and (2) a mechanism to build a collection element by element, which is the builder pattern. We will also present the Numeric pattern, which is used in methods like |sum| or |prod|.

\subsection{Function Encoding}

In Scala, it is common to use functions to manipulate collections. For example, in order to extract the positive numbers in a |List| of integers, we can use the |filter| method along with the following function:

\begin{lstlisting-nobreak}
 List(4,-2,1).filter(`x => x > 0`)
\end{lstlisting-nobreak}

However, since the Java Virtual Machine doesn't support functions (at least not until Java 7), Scala needs to provide a special translation for them:

\begin{lstlisting-nobreak}
 List(4,-2,1).filter({
     class $anon extends Function1[Int, Boolean] {
       def apply(x: Int): Boolean = `x > 0`
     }
     new $anon()
   })
\end{lstlisting-nobreak}

Still, the |Function1| is provided by the standard library and can't be overriden with a miniboxed version. Hence, in order to specialize functions, we need to provide our own function traits, which are miniboxed and perform the desugaring by hand.

This is done by creating a custom |MyFunc1| trait that receives two type parameters, |T| and |R|, which signal the argument and return of our function, i.e |(T => R)|. This trait exposes an abstract |apply| function that will contain the actual code of the function. Miniboxing is triggered by annotating both of the type parameters with |@miniboxed|:

\begin{lstlisting-nobreak}
 trait MyFunc1[`@miniboxed` -T, `@miniboxed` +S] {
   def apply(t: T): S
 }
\end{lstlisting-nobreak}

The plugin will generate five different traits, which will be used to encode functions. These correspond to the interface plus the 4 possible combinations for the 2 representations: (erased, erased), (erased, miniboxed), (miniboxed, erased), (miniboxed, miniboxed). The transformation will also create 4 versions of the |apply| method:

\begin{lstlisting-nobreak}
  abstract trait MyFunc1[-T, +R] extends Object {
    def `apply`(t: T): R
    def `apply_JL`(..., t: long): R
    def `apply_LJ`(..., t: R): long
    def `apply_JJ`(..., t: long): long
  }
\end{lstlisting-nobreak}

Then, just like methods, four different abstract traits that extend the previous interface will be created.

% Here is the optimized version for primitive type for both type parameters |T| and |S|:
% \begin{lstlisting-nobreak}
%   abstract trait `MyFunc1_JJ`[-Tsp, +Ssp]
%         extends MyFunc1[Tsp,Ssp] {
%     def $S_TypeTag(): Byte
%     def $T_TypeTag(): Byte
%     def apply(t: Tsp): Ssp
%     def apply_JJ(..., t: long): long
%     def apply_JL(..., t: long): Ssp
%     def apply_LJ(..., t: Tsp): long
%   }
%
% \end{lstlisting-nobreak}

Now, in order to express the previous function, we can write:

\begin{lstlisting-nobreak}
 new MyFunc1[Int, Boolean] { def apply(x: Int): Boolean = x > 0 }
\end{lstlisting-nobreak}

And the miniboxing transformation will translate this to:

\begin{lstlisting-nobreak}
 new `MyFunc1_JJ`[Int, Boolean] { ... }
\end{lstlisting-nobreak}

Now, any invocation of this function will actually invoke |apply_JJ|, thus completely avoiding boxing the primitive types |int| and |boolean| to their object representations.

% Note that only the |apply_JJ| is used in this case, since it determines the optimized version when both of the type parameters are primitive types. The others methods only forward to this method.
%
% Now, if one instantiates |MyFunc1[T,S]| with both primitive types for |T| and |S| and provides an implementation for the |apply| method, the class will actually override the |MyFunc1_JJ| abstract trait, and the implementation will be provided to |apply_JJ|. Since the function |apply_JJ| offers direct primitive types instead of references for arguments and return type, it will greatly enhance performance.

\subsection{Builder Pattern}

The Builder pattern is the key component necessary for collection comprehensions: It greatly reduces code duplication, since all the collection comprehensions reduce to creating a new collection with either transformed of filtered elements. It also brings flexibility, as shown by the following example:

\begin{lstlisting-nobreak}
scala> val map = Map(1 -> 2, 2 -> 3)
map: immutable.Map[Int,Int] = ...

scala> map.map(`{ case (x, y) => (y, x) }`)
res1: immutable.`Map[Int,Int]` = ...

scala> map.map(`{ case (x, y) => x }`)
res2: immutable.`Iterable[Int]` = ...
\end{lstlisting-nobreak}

% by specifying to such functions a Builder that describes a way to build the new data structure from these elements. (How)

% For instance, if one maps the elements of a |Map[Int, Int]| to some new combination |Long| |->| |Byte|, then the method should return a |Map[Double, String]|. But if one wants to map the key-value tuple to a single |Short| value, then it makes sense that the return type is no more a |Map|, but some |Iterable| collection, like a |List[Short]|. In other words:
%
% \begin{lstlisting-nobreak}
%  Map[Int,Int].map(f:(Int,Int) => (Long,Byte))
%       // returns a Map[Long,Byte]
%  Map[Int,Int].map(f:(Int,Int) => Short)
%       // returns an Iterable[Short]
% \end{lstlisting-nobreak}

To achieve this, the |map| function will rely on a |Builder| generated from the |CanBuildFrom| parameter, where |Repr| is the current collection and |That| is the resulting collection:

\begin{lstlisting-nobreak}
  def map[B, `That`](f: A => B)(implicit bf: CanBuildFrom[`Repr`, B, `That`]): That = {
    val b = bf(repr) // the builder
    for (x <- this)
      `b += f(x)`
    b.result
  }
\end{lstlisting-nobreak}



The Builder pattern also shows how type constructor polymorphism can play an essential role in factoring out boilerplate code without losing type safety \cite{adriaan}.

\subsection{Numeric Pattern}

Defining a generic type for a class can sometimes leads to inconvenient situations. Sometimes, one needs classes that depends only on numerical values. Since there's no common ancestor for numeric types (primitive or object), it's impossible to specify an adequate upper bound for a generic type that would directly offer mathematical operations.

The Numeric pattern solves this issue and allows to use custom functions on type parameters. This is done by creating a generic |Numeric| trait that provides mathematical operations for a certain type. By example, one could define a way to add two numerical values, by providing such a definition to the trait:

\begin{lstlisting-nobreak}
 trait Numeric[T] {
   def plus(x: T, y: T): T
   ...
 }
\end{lstlisting-nobreak}

If we specify the trait into different concrete extensions, we provide a concrete implementation that defines the addition between two values. We can write these specific classes for primitive numbers, e.g |Int| or |Float|, but we also could write it for a numeric class, such as |BigInteger|, as long as the definition is implemented. For instance, here's the code for the |Int| extension:

\begin{lstlisting-nobreak}
 implicit object Num_I extends Numeric[Int] {
   def plus(x: Int, y: Int): Int = x + y
   ...
 }
\end{lstlisting-nobreak}

Now, every time we want to use a type parameter as a numeric type, we enforce that the |Numeric| version of the type exists, so we can call the mathematical operations on them. Here is a complete example of a two-dimensional vector class:

\begin{lstlisting-nobreak}
 class Vec2D[T : Numeric](val x: T, val y: T) {
   def +(that: Vec2D[T]): Vec2D[T] = {
     val n = implicitly[Numeric[T]]
     new Vec2D[T](
       n.plus(this.x, that.x),
       n.plus(this.y, that.y))
   }
   ...
 }
\end{lstlisting-nobreak}

Since the |Numeric| implementations are likely to use primitive type parameters, boxing and unboxing would frequently occur. This is where the miniboxing specialization steps in. With a simple |@miniboxed| annotation on the type parameter of the |Numeric| class, a concrete extension would override an optimized version for primitive type. The classes that use the |Numeric| objects should also have a |@miniboxed| annotation. This would avoid every occurrence of boxing and unboxing, and enhance greatly the performance.
\section{Benchmarks}

In order to prove the efficiency of the miniboxing plugin, we wrote a benchmark about a common numerical application: the \textit{least squares method}. This method finds the best-fitting curve to a given set of points. Since it deals with numbers, type erasure would induce boxing and unboxing. The method should therefore run faster with the miniboxing plugin. The benchmark used is the following:

\begin{lstlisting-nobreak}
 // returns a random value between (-1,1)
 val random = new scala.util.Random(0)
 def rand = random.nextDouble - random.nextDouble
 
 val slope = 3.0  // arbitrary
 val intercept = 1.0  // arbitrary
 // func(x) = 3x + 1
 val func = new Function1[Int, Double] {
   def apply(x: Int): Double =
      slope*x + intercept
 }
 
 val size = 10000 // amount of points
 var xs: List[Double] = Nil // list of x-coord.
 var ys: List[Double] = Nil // list of y-coord.
 
 // fills xs/ys with function + noise
 var i = 0
 while (i < size) {
  xs = (i + rand) :: xs
  ys = (func(i) + rand) :: ys
  i += 1
 }
 
 val xy = xs.zip(ys) // list of (x,y) coord.
 
  // function (x, y) => x * y
 val fxy =
  new Function1[Tuple2[Double,Double], Double] {
     def apply(t: Tuple2[Double, Double]): Double = t._1 * t._2
 }
 
 // function x => x * x
 val fxx = new Function1[Double, Double] {
   def apply(x: Double): Double = x * x
 }
 
 val sumx  = xs.sum
 val sumy  = ys.sum
 val sumxy = listxy.map(fxy).sum
 val squarex = listx.map(fxx).sum
 
 // slope and intercept approximation
 val m = (size*sumxy - sumx*sumy) / (size*squarex - sumx*sumx)
 val b = (sumy*squarex - sumx*sumxy) / (size*squarex - sumx*sumx)
\end{lstlisting-nobreak}
%  
%  // was it a good approximation?
%  assert(Math.abs(m - slope) < 0.1)
%  assert(Math.abs(b - intercept) < 0.1)
% \end{lstlisting-nobreak}

If we run one version of the above benchmark with our mock-up collection, one with the plugin activated (|Miniboxed|) and one without (|Generic|), for different amount of points, we get the following results:

\begin{center}
\begin{tabular}{ |c|c|c| } 
 \hline
 Amount of points & |Generic| [cs] & |Miniboxed| [cs] \\ 
 \hline
 30,000 & 12.552 & 13.252 \\
 60,000 & 27.253 & 29.128 \\
 90,000 & 42.782 & 43.947 \\
 120,000 & 60.575 & 64.846 \\
 150,000 & 74.202 & 82.995 \\
 \hline
\end{tabular}
\end{center}

The results show that the erasure version runs $\sim$10\% faster. This is due to some issues causing boxing and unboxing to occur, and thus slowing down the execution. We are currently debugging on this; once solved, we would definitely have better numbers.
\section{Conclusions}

\bibliographystyle{abbrvnat}
\bibliography{main}

\end{document}
