\section{Miniboxing}

Miniboxing\cite{miniboxing-www} is a compilation scheme that improves the performance of generics in the Scala programming language. The miniboxing transformation is activated by annotating type parameters with |@miniboxed|, for both classes and methods:

\begin{lstlisting-nobreak}
class C[`@miniboxed T`](var t: T) {
  def foo(): T = t
}
\end{lstlisting-nobreak}

In order to understand the transformation behind this annotation, let's look at the following example. Assume we want to write a generic method that is likely to use a primitive type parameter:

\begin{lstlisting-nobreak}
def bar[`@miniboxed T`](t: T): T =
  (new C[T](t)).foo()
\end{lstlisting-nobreak}

Since the type parameter |T| of foo is marked as |@miniboxed|, the method will have two versions: an optimized version for value types, and the compatible, erasure-based, slow version of the method:

\begin{lstlisting-nobreak}
def bar_J(`T_Tag: byte`, t: `long`): `long` =
  (new C_J(t)).foo_J()   // optimized version
def bar(t: `Object`): `Object` =
  (new C(t)).foo()   // erasure-based version
\end{lstlisting-nobreak}

Every time the method |bar[T]| is called with a primitive type parameter, the optimized version |bar_J| will be used instead. However, when the method is called with an object type parameter, such as |String|, the original one will be used:

\begin{lstlisting-nobreak}
bar[String]("x")  // bar() is used
bar[Int](3)           // bar_J() is used
\end{lstlisting-nobreak}

Now let's look at the miniboxed class |C| from the first example. The same kind of transformation as the one for methods will occur, two representations of the same class will be made:

\begin{lstlisting-nobreak}
class C_J(`T_Tag: byte`, t: `long`) {
  def foo_J(): `long` = t
  def foo(): `long` = foo_J()
}

class C(t: `Object`) {
  def foo_J(): `Object` = foo()
  def foo(): `Object` = t
}
\end{lstlisting-nobreak}

Once again, if the class |C[T]| is instantiated with a primitive type paramater, it's |C_J| that will take place instead. Otherwise, it's the original |C| class that takes place. As one can see, the methods inside the class are wired in order to match the correct call. If one uses the method |foo| in the optimized |C_J|, it's actually the |foo_J| method that must be called:

\begin{lstlisting-nobreak}
val c_s = new C[String]("x")  // class C is used
val c_i = new C[Int](3)               // class C_J is used
println(c_s.foo())   // foo() is used
println(c_i.foo())   // foo_J() is used
\end{lstlisting-nobreak}

The process of specialization on classes brings two important things. First, it will specialize the fields of the class. This will improve performance, because the program now deals with a shorter representation of the values. This is done by referring the fields as long integers, instead of objects. The conversion between the actual primitive type and the long value stored will also be made in the backend.

Second, it will specialize the methods and their bodies in the class, just like stand-alone methods. This will also greatly improve performance. By specializing them, methods receive and return long integers, instead of references. Thanks to this, boxing and unboxing, originally occurring with references, will be avoided. Once again, conversion between the actual primitive type and the long value stored will be made in a hidden way.