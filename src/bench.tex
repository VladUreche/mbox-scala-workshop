\section{Benchmarks}

To benchmark performance, we implemented a mockup of the Scala collections library including |Traversable|, |Iterable|, |Iterator| and a linked list class with its builders. We also implemented other necessary parts of the Scala library, including |Function1|, |Tuple2| and the |Numeric| pattern. To asses the speedup of the miniboxing plugin used on the collections mockup, we implemented a common numerical application: the \textit{least squares method}. This method computes the parameters of a linear equation that best describes a given set of points. Since the benchmark deals with numbers, type erasure introduces boxing and unboxing operations. The method should therefore run faster with the miniboxing plugin. The benchmark used is the following:

\begin{lstlisting-nobreak}
 // list of (x,y) coordinates
 val xy = xs.zip(ys)

 // function (x, y) => x * y
 val fxy =
   new Function1[Tuple2[Double,Double], Double] {
     def apply(t: Tuple2[Double, Double]): Double = t._1 * t._2
   }

 // function x => x * x
 val fxx =
   new Function1[Double, Double] {
     def apply(x: Double): Double = x * x
   }

 // intermediary sums:
 val sumx  = xs.sum
 val sumy  = ys.sum
 val sumxy = listxy.map(fxy).sum
 val squarex = listx.map(fxx).sum

 // slope and intercept approximation:
 val m = (size*sumxy - sumx*sumy) / (size*squarex - sumx*sumx)
 val b = (sumy*squarex - sumx*sumxy) / (size*squarex - sumx*sumx)
\end{lstlisting-nobreak}
%
%  // was it a good approximation?
%  assert(Math.abs(m - slope) < 0.1)
%  assert(Math.abs(b - intercept) < 0.1)
% \end{lstlisting-nobreak}

If we run one version of the above benchmark with our mock-up collection, one with the plugin activated (|Miniboxed|) and one without (|Generic|), for different amount of points, we get the following results:

\begin{center}
\begin{tabular}{ |c|c|c| }
 \hline
 Amount of points & Generic [ms] & Miniboxed [ms] \\
 \hline
 500000 & 304 & 545 \\
1000000 & 1184 & 2177 \\
 \hline
\end{tabular}
\end{center}

The results show 44-45\% speedups when miniboxed collections are used, without any intervention on the programmer's side\footnote{The benchmarking code is identical for the two transformations, only the library contains miniboxed annotations.}.
%We expect we can further refine these numbers to up to 70\% speedup by tweaking the transformation.
It is also worth noting that scala specialization \cite{iuli-thesis}, the current solution used in the Scala compiler, was not able to successfully compile the example.

