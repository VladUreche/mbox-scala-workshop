\section{Scala Collections}

% \topic{Scala collections provide a high-level interface but lack in performance.}

\topic{In this paper we show how the miniboxing transformation enables improved collections}, which expose the same high-level interface without sacrificing performance. The next section presents the common patterns that enable the high-level interface in the Scala collections \cite{collections-arch}, and how miniboxing can be applied in order to improve performance.

%\topic{In this paper we show how the miniboxing transformation improves collection performance} while exposing the same high-level interface.
%Before we dive into optimizing, we will present the main patterns used to implement collections in Scala and how each of these patterns are transformed by the miniboxing transformation.
% Note that we will present only the most common patterns.
%Scala Collections use different patterns to expose their high-level interface consistently across all classes.

\subsection{Inheritance and Mixins}

\topic{Inheritance and mixins group the common behavior of different collections.} This reduces code duplication and gives rise to a convenient collection hierarchy, where each level of the inheritance makes more assumptions about the architecture than the previous level. For example, the path to a linked list goes through |Traversable|, |Iterable|, |Seq|, |LinearSeq| and finally |List|. % Each level makes an additional assumption: that the collection can be traversed, that it can be iterated, that it is a sequence, a linear sequence, and finally that it is a list.

This nesting and splitting of functionality makes is necessary to have deep miniboxing: Adding the |@miniboxed| annotation to a collection type parameter will not be enough to fully transform it, as most of its functionality will be inherited from parent traits. Instead, what needs to be done is to deeply visit all the parent traits and mark their arguments as |@miniboxed|:

\begin{lstlisting-nobreak}
 // trait/class definition needs to be marked:
 trait Traversable[`@miniboxed` +A] extends
 // parents' definitions also have to be marked:
       TraversableLike[`A`, Traversable[A]]
       with GenTraversable[`A`]
       with TraversableOnce[`A`]
       with GenericTraversableTemplate[`A`, Traversable] { ... }
\end{lstlisting-nobreak}

%Since the elements of the CollectionLike can be primitive types, the methods implemented can receive or return boxes of primitive values, which degrades performance. Thus, triggering the miniboxing plugin on the type parameter of the element offers better performance. However, we have to annote only the generic type of the elements contained in the collection, and not the type constructor. Indeed, since this is an abstraction over a container that will always be an object, there is no need to make a primitive type optimized version for it. Therefore, we get better performance if we use a custom CollectionLike with a |@miniboxed| notation on the type |T|:

Since the goal of Scala collections is to avoid code duplication, collection comprehensions, such as |map| and |filter|, all rely on a common mechanism: visiting each element in the collection, performing an action for it, and optionally adding a new element to the result collection. For example, |filter| visits all elements and for each element applies a predicate which decides whether the element should be part of the resulting collection or not.

The two key elements necessary for implementing collection comprehensions are: (1) the mechanism to visit each element using a custom function, which is implemented in |Traversable| and (2) a mechanism to build a collection element by element, which is the builder pattern. We will also present the Numeric pattern, which is used in methods like |sum| or |prod|.

\subsection{Function Encoding}

In Scala, it is common to use functions to manipulate collections. For example, in order to extract the positive numbers in a |List| of integers, we can use the |filter| method along with the following function:

\begin{lstlisting-nobreak}
 List(1,2,3).filter(`x => x < 0`)
\end{lstlisting-nobreak}

However, since the Java Virtual Machine doesn't support functions (at least not until Java 7), Scala needs to provide a special translation for them:

\begin{lstlisting-nobreak}
 List(1,2,3).filter({
     class $anon extends Function1[Int, Boolean] {
       def apply(x: Int): Boolean = `x < 0`
     }
     new $anon()
   })
\end{lstlisting-nobreak}

Still, the |Function1| is provided by the standard library and can't be overriden with a miniboxed version. Hence, in order to specialize functions, we need to provide our own function traits, which are miniboxed and perform the desugaring by hand.

This is done by creating a custom |MyFunc1| trait that receives two type parameters, |T| and |R|, which signal the argument and return of our function, i.e |(T => R)|. This trait exposes an abstract |apply| function that will contain the actual code of the function. Miniboxing is triggered by annotating both of the type parameters with |@miniboxed|:

\begin{lstlisting-nobreak}
 trait MyFunc1[`@miniboxed` -T, `@miniboxed` +S] {
   def apply(t: T): S
 }
\end{lstlisting-nobreak}

The plugin will generate five different traits, which will be used to encode functions. These correspond to the interface plus the 4 possible combinations for the 2 representations: (erased, erased), (erased, miniboxed), (miniboxed, erased), (miniboxed, miniboxed). The transformation will also create 4 versions of the |apply| method:

\begin{lstlisting-nobreak}
  abstract trait MyFunc1[-T, +R] extends Object {
    def `apply`(t: T): R
    def `apply_JL`(..., t: long): R
    def `apply_LJ`(..., t: R): long
    def `apply_JJ`(..., t: long): long
  }
\end{lstlisting-nobreak}

Then, just like methods, four different abstract traits that extend the previous interface will be created.

% Here is the optimized version for primitive type for both type parameters |T| and |S|:
% \begin{lstlisting-nobreak}
%   abstract trait `MyFunc1_JJ`[-Tsp, +Ssp]
%         extends MyFunc1[Tsp,Ssp] {
%     def $S_TypeTag(): Byte
%     def $T_TypeTag(): Byte
%     def apply(t: Tsp): Ssp
%     def apply_JJ(..., t: long): long
%     def apply_JL(..., t: long): Ssp
%     def apply_LJ(..., t: Tsp): long
%   }
%
% \end{lstlisting-nobreak}

Now, in order to express the previous function, we can write:

\begin{lstlisting-nobreak}
 new MyFunc1[Int, Boolean] { def apply(x: Int): Boolean = x > 0 }
\end{lstlisting-nobreak}

And the miniboxing transformation will translate this to:

\begin{lstlisting-nobreak}
 new `MyFunc1_JJ`[Int, Boolean] { ... }
\end{lstlisting-nobreak}

Now, any invocation of this function will actually invoke |apply_JJ|, thus completely avoiding boxing the primitive types |int| and |boolean| to their object representations.

% Note that only the |apply_JJ| is used in this case, since it determines the optimized version when both of the type parameters are primitive types. The others methods only forward to this method.
%
% Now, if one instantiates |MyFunc1[T,S]| with both primitive types for |T| and |S| and provides an implementation for the |apply| method, the class will actually override the |MyFunc1_JJ| abstract trait, and the implementation will be provided to |apply_JJ|. Since the function |apply_JJ| offers direct primitive types instead of references for arguments and return type, it will greatly enhance performance.

\subsection{Builder Pattern}

The Builder pattern is the key component necessary for collection comprehensions: It greatly reduces code duplication, since all the collection comprehensions reduce to creating a new collection with either transformed of filtered elements. It also brings flexibility, as shown by the following example:

\begin{lstlisting-nobreak}
scala> val map = Map(1 -> 2, 2 -> 3)
map: immutable.Map[Int,Int] = ...

scala> map.map(`{ case (x, y) => (y, x) }`)
res1: immutable.`Map[Int,Int]` = ...

scala> map.map(`{ case (x, y) => x }`)
res2: immutable.`Iterable[Int]` = ...
\end{lstlisting-nobreak}

% by specifying to such functions a Builder that describes a way to build the new data structure from these elements. (How)

% For instance, if one maps the elements of a |Map[Int, Int]| to some new combination |Long| |->| |Byte|, then the method should return a |Map[Double, String]|. But if one wants to map the key-value tuple to a single |Short| value, then it makes sense that the return type is no more a |Map|, but some |Iterable| collection, like a |List[Short]|. In other words:
%
% \begin{lstlisting-nobreak}
%  Map[Int,Int].map(f:(Int,Int) => (Long,Byte))
%       // returns a Map[Long,Byte]
%  Map[Int,Int].map(f:(Int,Int) => Short)
%       // returns an Iterable[Short]
% \end{lstlisting-nobreak}

To achieve this, the |map| function will rely on a |Builder| generated from the |CanBuildFrom| parameter, where |Repr| is the current collection and |That| is the resulting collection:

\begin{lstlisting-nobreak}
  def map[B, `That`](f: A => B)(implicit bf: CanBuildFrom[`Repr`, B, `That`]): That = {
    val b = bf(repr) // the builder
    for (x <- this)
      `b += f(x)`
    b.result
  }
\end{lstlisting-nobreak}



The Builder pattern also shows how type constructor polymorphism can play an essential role in factoring out boilerplate code without losing type safety \cite{adriaan}.

\subsection{Numeric Pattern}

Defining a generic type for a class can sometimes leads to inconvenient situations. Sometimes, one needs classes that depends only on numerical values. Since there's no common ancestor for numeric types (primitive or object), it's impossible to specify an adequate upper bound for a generic type that would directly offer mathematical operations.

The Numeric pattern solves this issue and allows to use custom functions on type parameters. This is done by creating a generic |Numeric| trait that provides mathematical operations for a certain type. By example, one could define a way to add two numerical values, by providing such a definition to the trait:

\begin{lstlisting-nobreak}
 trait Numeric[T] {
   def plus(x: T, y: T): T
   ...
 }
\end{lstlisting-nobreak}

If we specify the trait into different concrete extensions, we provide a concrete implementation that defines the addition between two values. We can write these specific classes for primitive numbers, e.g |Int| or |Float|, but we also could write it for a numeric class, such as |BigInteger|, as long as the definition is implemented. For instance, here's the code for the |Int| extension:

\begin{lstlisting-nobreak}
 implicit object Num_I extends Numeric[Int] {
   def plus(x: Int, y: Int): Int = x + y
   ...
 }
\end{lstlisting-nobreak}

Now, every time we want to use a type parameter as a numeric type, we enforce that the |Numeric| version of the type exists, so we can call the mathematical operations on them. Here is a complete example of a two-dimensional vector class:

\begin{lstlisting-nobreak}
 class Vec2D[T : Numeric](val x: T, val y: T) {
   def +(that: Vec2D[T]): Vec2D[T] = {
     val n = implicitly[Numeric[T]]
     new Vec2D[T](
       n.plus(this.x, that.x),
       n.plus(this.y, that.y))
   }
   ...
 }
\end{lstlisting-nobreak}

Since the |Numeric| implementations are likely to use primitive type parameters, boxing and unboxing would frequently occur. This is where the miniboxing specialization steps in. With a simple |@miniboxed| annotation on the type parameter of the |Numeric| class, a concrete extension would override an optimized version for primitive type. The classes that use the |Numeric| objects should also have a |@miniboxed| annotation. This would avoid every occurrence of boxing and unboxing, and enhance greatly the performance.